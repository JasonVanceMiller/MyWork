\documentclass[12pt]{article}



\usepackage[margin=1in]{geometry}  
\usepackage{graphicx}              
\usepackage{amsmath}               
\usepackage{amsfonts}              
\usepackage{amsthm}                
\usepackage{hyperref}
\usepackage{xcolor}

\newtheorem{thm}{Theorem}[section]
\newtheorem{lem}[thm]{Lemma}
\newtheorem{prop}[thm]{Proposition}
\newtheorem{cor}[thm]{Corollary}
\newtheorem{sol}[thm]{Solution}
\newtheorem{mydef}[thm]{Definition}

\begin{document}
\title{Math 124 Notes}
\author{Jason Miller}
\maketitle 

\tableofcontents

\newpage 
\section{Lecture 1, 9/24}
Beginning Topology by Sue E. Goodman will be closely followed, most resources are on their webpage, homework maybe physically submitted.  
\begin{itemize}
\item Chapter 1: Point set topology - lecture notes
\item Chapter 2: Surfaces 
\item Chapter 3: Euler characteristic, fundamental group, knot theory 

\end{itemize}
\subsection{Chapter 1, Point Set Topology}
Def: A metric space $X$ is a non empty set with a distance function $d, X*X 
\rightarrow \mathbb{R}^+_0$.\\
\begin{itemize}
\item There is a notion of symmetry, $d(x,y) = d(y,x)$
\item Triangle inequality, $d(x,z) \leq d(x,y) + d(y,z)$
\item $d(x,y) \longleftrightarrow x = y$ 
\end{itemize}

Some examples:
\begin{itemize}
\item Euclidean metric, aka $\mathbb{R}^n$ with the distance being the magnitude of the line between the points.
\item p-metric, $\mathbb{R}^n$, but distance is $(\sum_i |x_i - y_i|^p)^{\frac{1}{p}}$. 2-metric is the same as Euclidean metric. Called a norm in Numerical Analysis. 
\item Discrete metric, for any set $d(x,y) = 1 \longleftrightarrow x \neq y$. Like the path length in a unweighted graph. 
\end{itemize} 

$\varepsilon$-neighborhood in a metric space $(X,d)$. The set of points $y \in X$, called $B_\varepsilon (x) $ near $x$ such that $d(y,x) < \varepsilon$ and $0 < \varepsilon$.\\\\
2-metric gives circle neighborhoods, 1-metric gives diamond, $\infty$-metric gives a square. \\\\
For any $\emptyset \neq Y \subset X$ and $(X,d)$ is a metric space then $(Y,d)$ is a metric space.
\\\\
A point  $x \in X$ is called an interior point of $A \subseteq X$ if $\exists \varepsilon > 0$ such that $B_\varepsilon (x) \subseteq A$. The set of all interior points is $A^\circ$\\\\

A point $x \in X$ is called a limit point (cluster point, accumulation point) of $A$ if for $\forall \varepsilon > 0, \exists y \neq x$ such that $y \in B_\varepsilon(x) \cap A$, the set of all points is called $A'$. \\\\
A set is open if every point is interior, a set is closed if every limit point of the set belongs to A. 
\\\\
The closure of a set $\overline A = A \cup A'$
$$A \subseteq \overline{A}, A^\circ \subseteq A$$
\newpage
\section{Lecture 2 9/27}

For $(x,y)$ all points in the set are interior points, therefore it is a open set. 

The set of points $A = \{\frac 1 n | n \in \mathbb{N}\}$ is a peculiar example. $0$ is a limit point of the set. The set is not closed because $A' \not \subseteq A$. $A$ also has no interior points because you can always pick a $\varepsilon < \frac 1n - \frac{1}{n+1}$ around the point $\frac 1n$.  \\\\
If $0 \in A$ then the set would be closed. \\\\
$$\mathbb{Q}^\circ = \emptyset, \mathbb{Q}' = \mathbb{R}$$


\newpage 
\section{Lecture 3 9/29}

The product of closed sets are closed and the product of open sets are open. \\\\

Proof, suppose we have open points in $S_1$ and $S_2$. This implies that there are $\varepsilon_1$ and $\varepsilon_2$ neighborhoods around $P_1,P_2$. Then the point $P_1 \times P_2$ would have a $\varepsilon$ of the smaller of $\varepsilon_1, \varepsilon_2$.\\\\

Proof2, we have two closed sets $S_1,S_2$ and we want to know if $S_1 \times S_2$ is closed. 
Consider a limit point of $S_1 \times S_2$. This means that there is a $P_1 \times P_2$ that has a point $y$ that is arbitrarily close. Well that means that the components of $y$ are arbitrarily close to $P_1$ and $P_2$ in their respective sets. This means that $P_1$ and $P_2$ are in their sets from the closure and thusly $P_1 \times P_2$ is in $S_1 \times S_2$. 
\\\\
The $\varepsilon$ neighborhood around a open point is a open set. This can be done with triangle inequality to find a epsilon for each of the points. 











\end{document}