\documentclass[12pt]{article}



\usepackage[margin=1in]{geometry}  
\usepackage{graphicx}              
\usepackage{amsmath}               
\usepackage{amsfonts}              
\usepackage{amsthm}                
\usepackage{hyperref}
\usepackage{xcolor}

\newtheorem{thm}{Theorem}[section]
\newtheorem{lem}[thm]{Lemma}
\newtheorem{prop}[thm]{Proposition}
\newtheorem{cor}[thm]{Corollary}
\newtheorem{sol}[thm]{Solution}
\newtheorem{mydef}[thm]{Definition}
\usepackage{amsmath}

\begin{document}
\title{Math 124 Notes}
\author{Jason Miller}
\maketitle 

\tableofcontents

\newpage 
\section{Lecture 1, 9/24}
Beginning Topology by Sue E. Goodman will be closely followed, most resources are on their webpage, homework maybe physically submitted.  
\begin{itemize}
\item Chapter 1: Point set topology - lecture notes
\item Chapter 2: Surfaces 
\item Chapter 3: Euler characteristic, fundamental group, knot theory 

\end{itemize}
\subsection{Chapter 1, Point Set Topology}
Def: A metric space $X$ is a non empty set with a distance function $d, X*X 
\rightarrow \mathbb{R}^+_0$.\\
\begin{itemize}
\item There is a notion of symmetry, $d(x,y) = d(y,x)$
\item Triangle inequality, $d(x,z) \leq d(x,y) + d(y,z)$
\item $d(x,y) \longleftrightarrow x = y$ 
\end{itemize}

Some examples:
\begin{itemize}
\item Euclidean metric, aka $\mathbb{R}^n$ with the distance being the magnitude of the line between the points.
\item p-metric, $\mathbb{R}^n$, but distance is $(\sum_i |x_i - y_i|^p)^{\frac{1}{p}}$. 2-metric is the same as Euclidean metric. Called a norm in Numerical Analysis. 
\item Discrete metric, for any set $d(x,y) = 1 \longleftrightarrow x \neq y$. Like the path length in a unweighted graph. 
\end{itemize} 

$\varepsilon$-neighborhood in a metric space $(X,d)$. The set of points $y \in X$, called $B_\varepsilon (x) $ near $x$ such that $d(y,x) < \varepsilon$ and $0 < \varepsilon$.\\\\
2-metric gives circle neighborhoods, 1-metric gives diamond, $\infty$-metric gives a square. \\\\
For any $\emptyset \neq Y \subset X$ and $(X,d)$ is a metric space then $(Y,d)$ is a metric space.
\\\\
A point  $x \in X$ is called an interior point of $A \subseteq X$ if $\exists \varepsilon > 0$ such that $B_\varepsilon (x) \subseteq A$. The set of all interior points is $A^\circ$\\\\

A point $x \in X$ is called a limit point (cluster point, accumulation point) of $A$ if for $\forall \varepsilon > 0, \exists y \neq x$ such that $y \in B_\varepsilon(x) \cap A$, the set of all points is called $A'$. \\\\
A set is open if every point is interior, a set is closed if every limit point of the set belongs to A. 
\\\\
The closure of a set $\overline A = A \cup A'$
$$A \subseteq \overline{A}, A^\circ \subseteq A$$
\newpage
\section{Lecture 2 9/27}

For $(x,y)$ all points in the set are interior points, therefore it is a open set. 

The set of points $A = \{\frac 1 n | n \in \mathbb{N}\}$ is a peculiar example. $0$ is a limit point of the set. The set is not closed because $A' \not \subseteq A$. $A$ also has no interior points because you can always pick a $\varepsilon < \frac 1n - \frac{1}{n+1}$ around the point $\frac 1n$.  \\\\
If $0 \in A$ then the set would be closed. \\\\
$$\mathbb{Q}^\circ = \emptyset, \mathbb{Q}' = \mathbb{R}$$


\newpage 
\section{Lecture 3 9/29}

The product of closed sets are closed and the product of open sets are open. \\\\

Proof, suppose we have open points in $S_1$ and $S_2$. This implies that there are $\varepsilon_1$ and $\varepsilon_2$ neighborhoods around $P_1,P_2$. Then the point $P_1 \times P_2$ would have a $\varepsilon$ of the smaller of $\varepsilon_1, \varepsilon_2$.\\\\

Proof2, we have two closed sets $S_1,S_2$ and we want to know if $S_1 \times S_2$ is closed. 
Consider a limit point of $S_1 \times S_2$. This means that there is a $P_1 \times P_2$ that has a point $y$ that is arbitrarily close. Well that means that the components of $y$ are arbitrarily close to $P_1$ and $P_2$ in their respective sets. This means that $P_1$ and $P_2$ are in their sets from the closure and thusly $P_1 \times P_2$ is in $S_1 \times S_2$. 
\\\\
The $\varepsilon$ neighborhood around a open point is a open set. This can be done with triangle inequality to find a epsilon for each of the points. 

\newpage
\section{Lecture 4 10/1}
If $A$ and $B$ are open sets then $A \cap B$ and $A \cup B$ are open. Proof:\\\\
For $A \cap B$, consider a point $x$ in $A$ and $B$. Because these sets are open then there is a $\varepsilon$ neighborhood in both $A$ and $B$. Choosing the smaller of the two $\varepsilon$s we can find equal neighborhoods in both $A$ and $B$ which then exists in the intersection. Repeating this for all points means that $A \cap B$ is open. \\\\
For $A \cup  B$, any point belongs to either $A$ or $B$, and there is a $\varepsilon$ neighborhood in that respective set. This neighborhood is then in $A \cup B$. \\\\
A infinite union of open sets is still open, but a infinite intersection of open sets is not always open. For instance consider $\cap_{n = 1}^\infty (-\frac 1 n, \frac 1 n) = \emptyset$\\\\
For a metric space $(X,d)$ with $A \subseteq X$. $A$ is open iff $B = X \backslash A$ is closed. Proof: \\\\
Suppose that there was a point $x$ in $B'$ and not in $B$ making it not closed. This would mean that $x \in A$ meaning that there is a radius around $x$ not in $B$, making it not a limit point of $B$. \\\\
Suppose that there was a point $x$ in $A$ and not in $A^\circ$, making it not open. This means that there is a point not in $A$ that is arbitrarily close to $x$. But that point is in $B$ and that makes it a limit point of $B$. $B$ is closed therefore we have a contradiction. \\\\
Some basic set properties. $X \backslash (A \cup B)$ = $X \backslash A \cap X \backslash B$ and $X \backslash (A \cap B) = X \backslash A \cup X \backslash B$ which inherit the properties of infinite unions of open sets and finite intersection of open sets. This means that the finite union of closed sets is closed and the infinite intersection of closed sets is closed. \\\\
$(X \backslash A)^\circ = X \backslash \overline{A}$\\
$\overline{X \backslash A} = X \backslash A^\circ$\\\\
Definition of limit of a sequence. Limits are unique.


\newpage
\section{Lecture 5 10/4}
$$x \in A' \longleftrightarrow \exists \{x_n\}_{n=1}^\infty x_n \in A, x_n \neq x, x_n \longrightarrow x$$
$$x \in \overline{A} \longleftrightarrow \exists \{x_n\}_{n=1}^\infty x_n \in A, x_n \longrightarrow x$$
$x_n$ can equal $x$\\\\ 
$A$ is closed if for every sequnce the limit of that sequence is in $A$.\\\\
Two metrics, $d1$ and $d2$ are equivilent if $\forall x, \varepsilon > 0, \exists \delta > 0$ such that $B^2_\delta(x) \subseteq B^1_\varepsilon(x)$ and vice versa. This makes one set being open means the other is open.  

\newpage
\section{Lecture 6 10/6}
If two metric spaces are equivalent then the ideas of openness, closedness, limit points and convergence. \\\\
A topological space is a set $X \neq \emptyset$ togehter with a collection $\tau \subseteq P(X)$ called the topology of $X$. $$\emptyset, X \in \tau$$
$$A,B \in \tau \longrightarrow A \cap B \in \tau$$
$$A_n \in \tau \longrightarrow \bigcup A_n \in \tau$$
The sets in $\tau$ are called the open sets.\\\\
$\tau = \{\emptyset, X\}$ is called the trivial topoogy. \\\\
$b$ is called the basis of a topology if: \\
i) if for $B_1, B_2 \in b$, and $z \in B_1 \cap B_2$, then there exists $B \in b$ with $z \in B \subseteq B_1 \cap B_2$\\
ii) for every $x \in X$ there exists $B \in b$ with $x \in B$\\\\
THM: the union of all $B \in b$ is $\tau$






\end{document}