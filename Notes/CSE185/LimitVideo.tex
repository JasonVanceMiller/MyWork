\documentclass[11pt, twocolumn, landscape]{beamer}
%\usepackage[margin=1in]{geometry}  
%\usepackage{graphicx}              
\usepackage{amsmath}               
\usepackage{blindtext}
\usepackage{amsfonts}              
\usepackage{amsthm}                
\usepackage{xcolor}
%\usepackage{titletoc}
%\usepackage{pgfplots}
\usepackage{tikz}
\def\checkmark{\tikz\fill[scale=1](0,.35) -- (.25,0) -- (1,.7) -- (.25,.15) -- cycle;} 

%\usetikzlibrary{arrows}v
\usepackage{hyperref}
\hypersetup{
    colorlinks,
    citecolor=blue,
   filecolor=blue,
  linkcolor=blue,
   urlcolor=blue,
   linktoc=all
}




\newtheorem{thm}{Theorem}[section]
\newtheorem{lem}[thm]{Lemma}
\newtheorem{prop}[thm]{Proposition}	
\newtheorem{cor}[thm]{Corollary}
\newtheorem{sol}[thm]{Solution}
\newtheorem{mydef}[thm]{Definition}
\definecolor{dank}{RGB}{10,10,150}
\newcommand{\quest}[1]{\textcolor{dank}{#1}}
\newcommand{\com}[1]{\textcolor{blue}{#1}}


\title{What is Calculus?}


\setcounter{secnumdepth}{0} % sections are level 1


\begin{document}



\frame{\titlepage} 






\section{Approximating Length}
\frame{\frametitle{Approximating Length}


Image
%\includegraphics[scale = 0.5]{Fig0.jpg}


}


\frame{\frametitle{Approximating Length Attempt 1}
\begin{columns}[onlytextwidth]

\column{0.5 \textwidth}
Image
%\includegraphics[scale = 0.3]{Fig0.jpg}
\column{0.5 \textwidth}
\begin{itemize}
\item
Our first attempt is to approximate $\sin(x)$ with 5 lines. \pause
\item
Approximate value is 3.9656 \pause
\item
But there are some obvious errors
\end{itemize}
\end{columns}
}

\frame{\frametitle{Approximating Length Attempt 2}
\begin{columns}[onlytextwidth]

\column{0.5 \textwidth}
Image
%\includegraphics[scale = 0.3]{Fig2.jpg}\\
%\includegraphics[scale = 0.3]{Fig3.jpg}
\column{0.5 \textwidth}

\begin{itemize}
\item
Our second attempt is to use 19 lines. \pause
\item
Approximate value is 4.1707 \pause
\item
The errors are less pronounced, but are still present. \pause
\item
What if we could approximate the length with lines of zero length 
\end{itemize}

\end{columns}
}



\section{Zero Distance} 
\frame{\frametitle{The Notion of Zero Distance} 





\begin{align*}
\frac{x^3-2x^2}{x - 2}& \hspace{10 mm} x \neq 2 \\\\ 
\frac{x^2(x-2)}{(x - 2)}& \hspace{10 mm} x \neq 2 \\\\ 
x^2& \hspace{10 mm} x \neq 2
\end{align*}



}

\frame{\frametitle{What Does This Look Like}


image
%\includegraphics[scale = 0.5]{Fig4.jpg}


}


\frame{\frametitle{How Do We Describe This}
\begin{columns}[onlytextwidth]

\column{0.5 \textwidth}
image
%\includegraphics[scale = 0.3]{Fig4.jpg}

\column{0.5 \textwidth}

\begin{itemize}
\item
"If $x = 2$ was in the domain then $f(2)$ would be 4" \pause
\item
\textcolor{red}{But $x = 2$ is not in the domain, so this is a vacuous statement.}
\end{itemize}

\end{columns}
}

\frame{\frametitle{How Do We Describe This}
\begin{columns}[onlytextwidth]

\column{0.5 \textwidth}
image
%\includegraphics[scale = 0.3]{Fig4.jpg}

\column{0.5 \textwidth}

\begin{itemize}
\item
"The values of $f(x)$ around $x = 2$ have a value of 4" \pause
\item
\textcolor{red}{1.9 is near 2 but it has a value of $f(1.9)$ is 3.61 which isn't 4}
\end{itemize}

\end{columns}
}

\frame{\frametitle{How Do We Describe This}
\begin{columns}[onlytextwidth]

\column{0.5 \textwidth}
image
%\includegraphics[scale = 0.3]{Fig4.jpg}

\column{0.5 \textwidth}

\begin{itemize}
\item
"As $x$ get \textit{arbitrarily} close to 2, the value of the $f(x)$ gets \textit{arbitrarily} close to 4" \pause
\item
\textcolor{green}{\checkmark} \pause 
\item
\[\lim_{x \to 2} \frac{x^3-2x^2}{x-2} = 4\]
\end{itemize}

\end{columns}
} 

\frame{\frametitle{Revisiting the Length Problem}
\begin{columns}[onlytextwidth]

\column{0.5 \textwidth}
image
%\includegraphics[scale = 0.3]{Fig1.jpg}\\
%\includegraphics[scale = 0.3]{Fig2.jpg}
\column{0.5 \textwidth}

\begin{itemize}
\item
What we saw previously is that as the step size gets smaller, the approximation becomes more and more accurate. \pause
\item
It turns out that we can look at the limit as the step size approaches zero. \pause
\item
\[\lim_{\text{step size} \to 0} \text{approx len} = \text{true len}\]\pause
\item
$\int_0^1 \sqrt{1 + 4\pi^2cos^2(2\pi x)} dx$
\end{itemize}

\end{columns}
} 



\frame{\frametitle{So What is Calculus}
\begin{itemize}
\item
As we have seen, calculus is all about trying to solve complex problems with infinitely many or infinitely small quantities through the use of clever manipulation and cancellation. \pause
\item
Some applications of calculus are: building accurate calculators, finding the orbits of planets, determining the volume and surface area of any object, modeling populations of animals in the wild, and much much more.

\end{itemize}


}


\end{document}
